\documentclass[11pt]{report}
\usepackage[usenames,dvipsnames]{color}
\usepackage{listings}
\title{TFE4170 Assignment 4}
\author{Ole Magnus Ruud \\ Vegar K\aa sli}
\begin{document}

\maketitle
\clearpage

\section*{Introduction}

All code is included in the attached archive. 

\section*{Task 1}

We divided the memory layout into two parts: A static part containing all the
status words for the different modules and freeloc, and a circular buffer for
all the packets. The layout is illustrated in table \ref{table}. 


\begin{table}[htbp]
  \centering
  \begin{tabular}{|l|l|}
    \hline
    Address&Contents \\
    \hline
0x000 & Control status word \\
0x004 & freeloc \\
0x008 - 0x044 & Adapter/Button status words \\ 
0x048 &Start of circular buffer\\
0xFFF &End of circular buffer\\
    \hline
  \end{tabular}
  \caption{Addresses in FastMem (byte-indexed)}
  \label{table}
\end{table}
\section*{Task 2}

The adapter functionality was split into two threads, one listening for
changes to the button status in memory, and one responding to button pushes.
The adapter had to implement the bus interface from the previous assignment
to be compatible with the buttons. The functionality of the inherited push
method was reduced to notifying the button push thread(pushListen).

The thread waiting for changes to the status in memory(main) simply reads the 
corresponding status location at set intervals, and calls the light function
of the button whenever it is necessary. It does not start reading from memory
until notified by pushListen, as the light will never be lit before the button
has been pushed.

The pushListen thread generates a new packet when notified, and uses the send
method to write it to the bus. It also calls a bounce method that generates 
one or more releasepackets (task 5).

The send method is pretty much implemented as specified in the assignment text.
The ring buffer pointer is checked to be suitable to the packet size and set to 
the base buffer address if the end is reached. First the basic packet is sent, 
and in case of a bigger packet (task 5) the rest is sent in a separate burst.

\section*{Task 3}
The basic functionality of the control is kept the same as in exercise two, but
the communication with the other modules had to be adapted to fit the new bus
model.

Control checks its status word for a non-zero value, and if it reads one, it
will split it into two parts: The address in FastMem for the packet, and the
identity of the button/adapter from which the packet originated. It will then
read from FastMem, at the location pointed to in the status word.

If the packet in FastMem appears to be valid, the Control module will see if
the button that was pressed, is in fact the correct button in the sequence. If
it was correct, it will advance its internal counter to the next button, and
write to that button's status word in FastMem; setting it to a non-zero value.
If it was not the correct button, it will write to all the previously pressed
buttons' status words, setting them to zero.

Finally it will set its status word to zero. It then sleeps for a short 
while, before it again checks to see if its status word is set to any non-zero 
value.

\section*{Task 4}
The only difference in the top module from exercise 2, is the change of bus.
SimpleBus is more complex than the naive bus in exercise 2, and requires a
sc\_clock instance to clock the bus. The period specified in the sc\_clock needs
to be much less than the Tr, which the exercise specifies. This is so that the
bus can manage the communication before a new button is pressed.

With a clock period of 10us, the same testbed from exercise 2 produces
correct result. 

\section*{Task 5}

We did not manage to break the system by just doing as specified in the assignment.
As TR was reduced, the number of sent packages was also reduced, as only one word
could be sent by the bus per bus cycle, and the timeslot for sending packets 
was also constrained by TR.

As far as we could understand the purpose
was to make the system break by writing enough release packets so that the push
packet was overwritten. As the buffer size was around 1000 words, we had to make
the bus speed at least 1000 times faster than the control module to be able to
overwrite a push packet before it could be read. Another requirement was  
TR being smaller than the control module delay, and big enough so that we could
send enough packets in the 3TR timeslot. 

By setting the bus clock period to 10 us, the control sleep time to 50 ms and
TR to around 10 ms the system was effectively broken with some random delays between
the release packets included. These numbers might not be that sensible, but at least
we managed to break it. In our search for system weaknesses we also learned quite 
a lot about the bus model. 

\end{document}
