\documentclass[11pt]{report}
\usepackage[usenames,dvipsnames]{color}
\usepackage{listings}
\title{TFE4170 Assignment 4}
\author{Ole Magnus Ruud \\ Vegar K\aa sli}
\begin{document}

\maketitle
\clearpage

\section*{Introduction}


\section*{Task 1}
We divided the memory layout into two part: A static part containing all the
status words for the different modules and freeloc, and a circular buffer for
all the packets. 
\\
\\Addresses in FastMem (byte-indexed)
\\0x000 Control status word
\\0x004 freeloc
\\0x008 - 0x044 Adapter/Button status words
\\0x048 Start of circular buffer
\\0xFFF End of circular buffer

\section*{Task 2}

\section*{Task 3}
The basic functionality of the control is kept the same as in exercise two, but
the communication with the other modules had to be adapted to fit the new bus
model.\\
\\Control checks its status word for a non-zero value, and if it detects one, it
will split it into two parts: The address in FastMem for the packet, and the
identity of the button/adapter from which the packet originated. It will then
read from FastMem, at the location pointed to in the status word, while also
acquiring a lock on the SimpleBus, so that subsequent writes are guaranteed not
to block.\\
\\If the packet in FastMem appears to be valid, the Control module will see if
the button that was pressed, is in fact the correct button in the sequence. If
it was correct, it will advance its internal counter to the next button, and
write to that button's status word in FastMem; setting it to a non-zero value.
If it was not the correct button, it will write to all the previously pressed
buttons' status words, setting them to zero. Non the less it will still obtain
the lock on the bus.\\
\\Finally it will set it status word to zero, and release the lock. It then
sleeps for a short while, before it again checks to see if its status word is
set to any non-zero value.

\section*{Task 4}


\end{document}
